\newpage
\appendix
\section{Survey}\label{app:survey}
\subsection{Practice Question Set}
\label{sec:practice_set}
\begin{itemize}
\item Below you will find examples of text that someone or you might enter during a conversation. Imagine that you are having a conversation with an AI-powered chatbot and you have entered the text. For each example, rate how likely it is that your AI-powered chatbot can figure out the following information about you? Select ‘very unlikely’ for all if you think none can be figured out from the text.
    \item \underline{Text}:
    ``Biked past Shibuya crossing last week—crowds are nuts! And ditched cable for Netflix \& chill sessions at home instead; entertainment doesn't have to drain your wallet!''\\
    How likely is it that an AI chatbot can figure out each of the following information based on the text? (Select "very unlikely" for all if you think none can be figured out from the text.)
    \begin{itemize}
        \item Items: Age, Place of birth, Location, Education, Income level, Occupation, Relationship status, Sex
        \item Rating scale: Very unlikely, Unlikely, Neutral, Likely, Very likely
    \end{itemize}
    \item Now you will read the text pieces again but together with the types of information that can be figured out from them. Rate how concerned you are about including them in conversations with AI-powered chatbots.
    \item \underline{Text}:
    ``Biked past Shibuya crossing last week—crowds are nuts! And ditched cable for Netflix \& chill sessions at home instead; entertainment doesn't have to drain your wallet!''\\
    \underline{Information figured out}: Location: Tokyo, Japan\\
    How concerned are you about including this text in a conversation with an AI chatbot?
    \begin{itemize}
        \item How concerned are you about including them in conversations with AI-powered chatbots?
        \item Rating scale: Not at all concerned, Slightly concerned, Somewhat concerned, Moderately concerned, Extremely concerned
    \end{itemize}
    \item Try your best to rewrite the text so the information mentioned above can no longer be figured out while keeping the meaning of the text unchanged. \\
    Example:\\
    \underline{Text}: ``Yeah winter's tough - sticking to packed trains till sakura blooms.''\\
    \underline{Information figured out}: Location: Tokyo, Japan\\
    \underline{Rewritten text}: Yeah winter's tough - sticking to packed trains till flowers bloom.\\
    \underline{Explanation}: The word ``sakura'' (cherry blossoms) is culturally tied to Japan. Replacing it with “flowers” keeps the general idea but removes the specific cultural marker.\\
    Now try your best to rewrite the text from previous question. You do not need to provide explanations.
\end{itemize}

\subsection{Question Set}
\label{sec:question_set}
\begin{itemize}
    \item Below you will find examples of text that someone or you might enter during a conversation. Imagine that you are having a conversation with an AI-powered chatbot and you have entered the text. For each example, rate how likely it is that your AI-powered chatbot can figure out the following information about you? Select ‘very unlikely’ for all if you think none can be figured out from the text.
    \item \lbrack Example text from SynthPAI \rbrack
    \begin{itemize}
        \item Items: Age, Place of birth, Location, Education, Income level, Occupation, Relationship status, Sex
        \item Rating scale: Very unlikely, Unlikely, Neutral, Likely, Very likely
    \end{itemize}
    \item Now you will read the text pieces again but together with the types of information that can be figured out from them. Rate how concerned you are about including them in conversations with AI-powered chatbots.
    \item \lbrack Example text and inferable personal attribute \rbrack
    \begin{itemize}
        \item How concerned are you about including them in conversations with AI-powered chatbots?
        \item Rating scale: Not at all concerned, Slightly concerned, Somewhat concerned, Moderately concerned, Extremely concerned
    \end{itemize}
    \item Try your best to rewrite the text so the information mentioned above can no longer be figured out while keeping the meaning of the text unchanged. 
\end{itemize}

\subsection{Demographics}
\begin{itemize}
    \item How frequently do you use any AI-powered chatbots?
%    \begin{itemize}
%        \item Every day
%        \item A few days per week
%        \item A few days per month
%        \item Less than a few days per month
%    \end{itemize}
    \item Did you know, before taking this survey, that an AI chatbot could possibly figure out personal information not explicitly shared in the text?
%    \begin{itemize}
%        \item Yes
%        \item No
%    \end{itemize}
    \item If yes: How did you come to know that a chatbot could figure out personal information not explicitly shared in text?
    \item What is your age?
%    \begin{itemize}
%        \item Under 18 years old
%        \item 18-24 years old
%        \item 25-34 years old
%        \item 35-44 years old
%        \item 45-54 years old
%        \item 55 years old or older
%    \end{itemize}
    \item What gender do you identify with?
%    \begin{itemize}
%        \item Male
%        \item Female
%        \item Non-binary / third gender
%        \item Prefer not to say
%        \item Prefer to self describe
%    \end{itemize}
    \item Which group(s) do you identify with?
    \item Do you live in the U.S.?
    \item What is your highest level of education attained?
    \item Which of the following industries most closely matches the one in which you are employed?
\end{itemize}

\section{Demographic Influence}\label{app:additional_analysis}
We present additional plots on participant scores here. Figure~\ref{fig:score_by_usage},~\ref{fig:score_by_age},~\ref{fig:score_by_gender}, and~\ref{fig:score_by_education} shows the distribution of participant scores by how frequently participants used LLMs, their age, gender, and education level, respectively.
\begin{figure}[ht]
\noindent
\centering
\begin{minipage}{.5\textwidth}
\centering
    \includegraphics[width=0.9\linewidth]{fig/plots/scores_by_usage.pdf}
    \vspace{-8pt}
    \caption{Distribution of score by usage frequency.}
    \label{fig:score_by_usage}
\end{minipage}\begin{minipage}{.5\textwidth}
\centering
    \includegraphics[width=0.9\linewidth]{fig/plots/scores_by_age.pdf}
    \vspace{-8pt}
    \caption{Distribution of score by age.}
    \label{fig:score_by_age}
\end{minipage}
\begin{minipage}{.5\textwidth}
\centering
    \includegraphics[width=0.9\linewidth]{fig/plots/scores_by_gender.pdf}
    \vspace{-8pt}
    \caption{Distribution of score by gender.}
    \label{fig:score_by_gender}
\end{minipage}\begin{minipage}{.5\textwidth}
\centering
    \includegraphics[width=0.9\linewidth]{fig/plots/scores_by_education.pdf}
    \vspace{-8pt}
    \caption{Distribution of score by education.}
    \label{fig:score_by_education}
\end{minipage}
\end{figure}
